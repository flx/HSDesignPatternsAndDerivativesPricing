\let\origlinebreak=\\
\renewcommand{\sectfont}{\bf}
\long\def\ignore#1{}
%let framed = False
\usepackage[english]{babel}
\usepackage{hyperref}
\usepackage{mathpazo}
\usepackage{colortbl}
\usepackage{calc}
\usepackage{paralist}
\usepackage{ifthen}
\usepackage{relsize}
\usepackage{xspace}
\usepackage{tabularx}
%if framed
\usepackage{framed}
\FrameSep=2\fboxsep
%endif
\newcommand*{\PDF}{{\smaller{PDF}}\xspace}
\newcommand*{\CTAN}{{\smaller{CTAN}}\xspace}
\setdefaultitem{\textbf{--}}{}{}{}
\let\defined\textbf
%let doc = True
%include lhs2TeX.fmt
\newlength{\lwidth}
\newlength{\cwidth}
\setlength{\lwidth}{0pt}
\setlength{\cwidth}{0pt}
%separation 2
%latency 2
\let\origcolor=\color
\newcommand{\dep}[1]{{\origcolor{red}#1}}
\def\swgt#1{\switch[\value{step}>#1]}%
\def\ro#1{\ifthenelse{\value{step}=#1}{\origcolor{red}}{}}%
%hyperref needs some setup, especially after pdfscreen
\hypersetup{%
  colorlinks=True,%
  pdfmenubar=True,%
  pdfcenterwindow=False,% 
  pdffitwindow=False}%
\newcounter{pagesave}
% This one is with color:
%subst code a = "\begin{colorcode}'n" a "\end{colorcode}\resethooks'n" 
\definecolor{rlcolor}{gray}{.8}
\arrayrulecolor{rlcolor}
\definecolor{hcolor}{gray}{.7}
\newenvironment{colorcode}{%
  \colorsurround
  \(%
  \pboxed\SaveRestoreHook}{%
  \ColumnHook\endpboxed
  \)%
  \endcolorsurround}
\newenvironment{colorsurround}{\colorverb}{\endcolorverb}
\newenvironment{colorarray}{%
  \colorsurround
  \(%
  \array}{%
  \endarray
  \)%
  \endcolorsurround}
\makeatletter
\newenvironment{colorverb}{%
  \parskip=\abovedisplayskip\par\noindent
  \begingroup\small% small changes displayskips!
%if color
  \tabular{@@{}>{\columncolor{codecolor}}p{\linewidth}@@{}}%
%elif framed
  \framed
%else
  \tabular{@@{}||p{\linewidth-2\arraycolsep-2\arrayrulewidth-2pt}||@@{}}%
  \hline \\[-1.5ex]
  \let\myendofline=\\
%endif
  \let\\=\origlinebreak}{%
%if not color && not framed
  \myendofline[.5ex]\hline
%endif
%if framed
  \endframed
%else
  \endtabular
%endif
  \endgroup
  \parskip=\belowdisplayskip\par\noindent
  \ignorespacesafterend}
\makeatother
\newenvironment{important}[1][Important]%
  {\colorsurround
   \centering
   \bfseries\textsc{#1:}\ }%
  {\endcolorsurround}

%\definecolor{codecolor}{rgb}{.982, .902, .902}% original
%\definecolor{codecolor}{rgb}{1,.898,.667}% so'n orange
\definecolor{codecolor}{rgb}{1,1,.667}
%format forall(a) = "\forall " a "\relax"